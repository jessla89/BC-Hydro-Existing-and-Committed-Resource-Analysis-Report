\documentclass{report}
\usepackage{graphicx}
\usepackage{hyperref}
\usepackage[left=2cm,right=2cm,top=2cm,bottom=1.5cm]{geometry}
\usepackage{booktabs}
\usepackage{array}
\usepackage{caption}
\usepackage[list=true,listformat=simple]{subcaption}
\usepackage{multirow}
\makeatletter
\setlength\@fptop{0pt}             % default: '0\p@ \@plus 1fil'
\setlength\@fpsep{2\baselineskip}  % default: '8\p@ \@plus 2fil'
\makeatother

% Title Page
\title{Resources \\ \small Analysis of Existing and Future Resources available to meet B.C.'s Electricity needs}
\author{Jessla Varaparambil Abdul Kadher}
\begin{document}
\maketitle

\tableofcontents
\listoffigures
\listoftables

 
\chapter{Existing Resources}

\section{Introduction} 
\par BC Hydro meets the demand of its customers by relying on its heritage resources as well as purchasing power from independent power producers (IPPs) and transmission owners or providers. An existing supply-side resource includes BC Hydro’s Heritage hydroelectric and thermal (natural gas-fired) generating resources, as well as independent power producer (IPP) facilities delivering electricity to BC Hydro. Whereas committed supply-side resources are resources for which material regulatory and BC Hydro executive approvals have been secured. Majority of BC Hydro’s customers are connected to BC Hydro’s integrated system, which is an interconnected network of transmission lines, distribution lines and substations linking generating stations to one another and to customers throughout BC Hydro’s service area. Some of BC Hydro customers live in remote areas that are not served by the integrated system. These are the isolated customers who are connected to free-standing generating facilities and the local generation serves these Non-Integrated Areas (NIAs).  
\subsection{BC Hydro Heritage Resources}
\par BC Hydro relies mainly on the power of moving or falling water to produce mechanical/electrical energy to power the province of B.C. \href{https://www.bchydro.com/energy-in-bc/operations/our-facilities.html#:~:text=We%20have%2031%20generating%20stations%20sending%20electricity%20along%2075%2C000%20kilometres%20of%20power%20lines%20that%20range%20over%20mountain%20tops%20and%20river%20valleys%2C%20forested%20wilderness%20and%20high%20density%20urban%20areas.}
{BC Hydro has 31 Hydroelectric and 2 thermal generating stations sending electricity along 75,000 kilometres of power lines that range over mountain tops and river valleys, forested wilderness and high density urban areas.}
\par

The integrated and non-integrated resource capacity distribution for BC Hydro heritage resources are as shown in Figure~\ref{fig:capBCH}.


\begin{figure}[h]
\begin{subfigure}[b]{0.5\textwidth}
   \centering 
   \includegraphics[height=2in]{bchi}
   \caption{Integrated} 
   \label{fig:bchi}
\end{subfigure}% 
\begin{subfigure}[b]{0.5\textwidth}
   \centering 
   \includegraphics[height=2in]{bchnon}
   \caption{Non-integrated} 
   \label{fig:bchnon}
\end{subfigure}%
   \caption{Capacity of Integrated BC Hydro Generating Stations by Region} 
   \label{fig:capBCH}
\end{figure}


A detailed breakdown of integrated and non-integrated BC Hydro heritage resources by region and type is given in Table \ref{tab:capBCH} and Table \ref{tab:capBCHNon}. 

 \begin{table}[ht]
\centering 
 \begin{tabular}{ |c|c|c|  }
 \hline
 &  &  \\
 \textbf{Region}&\textbf{Type of Generation}&\textbf{Capacity(MW)}\\ [0.5ex]
 \hline
 & & \\[-1ex]
\textbf{Vancouver Island} & Hydroelectric & 469.0 \\[2ex]

 \multirow{2}{*}{\textbf{Peace Region}}
     & Hydroelectric & 4815.4 \\
     & Thermal & 73.0 \\[2ex]

 \multirow{2}{*}{\textbf{North Coast }}
   & Thermal & 46.0 \\
& Hydroelectric & 7.0 \\[2ex]

\textbf{Lower Mainland} & Hydroelectric & 1118.7 \\[2ex]

\textbf{Columbia Region} & Hydroelectric & 6728.5 \\[2ex]
  \hline
\end{tabular}
 \caption{Capacity of Integrated resources by type and region} 

\label{tab:capBCH}
\end{table}

\begin{table}
 \centering 
 
 \begin{tabular}{ |c|c|c|  }
 \hline
 &  &  \\
 \textbf{Region}&\textbf{Type of Generation}&\textbf{Capacity(MW)}\\ [0.5ex]
 \hline
 & & \\[-1ex]
\textbf{North Coast} & Diesel & 46.8 \\[2ex]
\textbf{Peace Region} & Diesel & 8.0 \\[2ex]
\textbf{North Coast} & Hydroelectric & 2.0 \\[2ex]
\textbf{Vancouver Island} & Diesel & 0.3 \\ [0.5ex]
 \hline
 \end{tabular}
 \caption{Capacity of Non-Integrated resources by type and region} 
\label{tab:capBCHNon}
\end{table}


\subsection{Independent Power Producer Projects}
To support BC's energy needs as they change and grow over time, BC Hydro purchases power from independent power producers (IPPs). IPPs develop and operate projects using resources such as wind, water and biomass and can include power production companies, municipalities, and Indigenous Nations. As of April 1, 2025, BC Hydro has 119 Electricity Purchase Agreements (EPAs) with Independent Power Producers. In total, these projects are capable of delivering approximately 18,662 gigawatt hours of annual supply and approximately 5,298 megawatts of capacity.

The integrated and non-integrated resource capacity distribution for IPP resources by region are as shown in Figure~\ref{fig:capIPP}.

\begin{figure}[h]
\begin{subfigure}[b]{0.5\textwidth}
   \centering 
   \includegraphics[height=2in]{iipp}
   \caption{Integrated} 
   \label{fig:iipp}
\end{subfigure}% 
\begin{subfigure}[b]{0.5\textwidth}
   \centering 
   \includegraphics[height=2in]{ippnon}
   \caption{Non-integrated} 
   \label{fig:ippnon}
\end{subfigure}%
   \caption{Capacity of IPP Resources by Region} 
   \label{fig:capIPP}
\end{figure}

A detailed breakdown of integrated and non-integrated IPP resources by region and type is given in Table \ref{tab:capIPP} and Table \ref{tab:capIPPNon}. 

 \begin{table}[h!]
\centering 
 \begin{tabular}{ |c|l|c|}
 \hline
 &  &  \\
 \textbf{Region}&\textbf{Type of Generation}&\textbf{Capacity(MW)}\\ [0.5ex]
 \hline
 & & \\[-1ex]
\multirow{7}{*}{\textbf{Vancouver Island}}
& Gas Fired Thermal & 275.0 \\
& Wind & 201.2 \\
& Non-Storage Hydro & 93.4 \\
& Biomass & 55.0 \\
& Storage Hydro & 29.9 \\
& Biogas & 0.355 \\
& ERG & 0.3 \\[2ex]

 \multirow{5}{*}{\textbf{Peace Region}}
 & Wind & 515.8 \\
& Biomass & 303.5 \\
& Gas Fired Thermal & 120.0 \\
& Non-Storage Hydro & 20.5 \\
& ERG & 6.0 \\[2ex]

 \multirow{3}{*}{\textbf{North Coast }}
 & Storage Hydro & 934.2 \\
& Non-Storage Hydro & 298.8 \\
& Biomass & 52.0 \\[2ex]

 \multirow{5}{*}{\textbf{Lower Mainland}}
& Non-Storage Hydro & 1046.4 \\
& Biomass & 112.0 \\
& MSW & 24.8 \\
& Storage Hydro & 10.9 \\
& Gas Fired Thermal & 8.8 \\[2ex]

\multirow{7}{*}{\textbf{Columbia Region}}
& Non-Storage Hydro & 509.1 \\
& Storage Hydro & 306.1 \\
& Biomass & 294.9 \\
& Wind & 30.0 \\
& ERG & 16.5 \\
& Biogas & 4.8 \\
& Solar & 2.09 \\[2ex]

\hline
\end{tabular}
\caption{Capacity of Integrated IPP resources by type and region} 
\label{tab:capIPP}
\end{table}

\begin{table}[h!]
 \centering 
 \begin{tabular}{ |c|c|c|}
 \hline
 &  &  \\
 \textbf{Region}&\textbf{Type of Generation}&\textbf{Capacity(MW)}\\ [0.5ex]
 \hline
 & & \\[-1ex]
\textbf{Peace Region} & Biomass & 0.5 \\[2ex]
\multirow{2}{*}{\textbf{North Coast}} 
& Non-Storage Hydro & 15.0 \\
& Storage Hydro & 11.0 \\
 \hline
 \end{tabular}
 \caption{Capacity of Non-Integrated IPP resources by type and region} 
\label{tab:capIPPNon}
\end{table}
\section{Conclusion} 
This chapter explored the existing resources available in B.C. to meet the electricty demand of british columbians. BC Hydro heritage and IPP resources capacities were analysed based on the type and location of the resource. 

\chapter{BC Hydro Heritage Resources}
\section{Introduction}
BC Hydro relies mainly on the power of moving or falling water to produce mechanical/electrical energy to power the province of B.C.
There are 31 grid-tied hydroelectric and 2 thermal generating stations in British Columbia operated by BC Hydro. In addition there are also 17 Non-Integrated Generating stations operated by BC Hydro. This chapter explores each of these generating facilities by Region in which they are located. BC Hydro operating regions can be classified into 5 regions namely Lower Mainland, Vancouver Island, Peace Region, Columbia Region, and the North Coast. While the 33 grid connected generating stations are distributed in all the 5 regions, the 17 non-integaretd generating stations are located in Vancouver Island, North Coast and Peace region only. 
The Grid connected generating stations in B.C. are predominantly Hydroelectric with the 2 Thermal Genarting stations located 1 each in Peace region and North Coast. out of the 17 non-integrated generating stations, 16 are diesel generating station while the lone Hydroelectric generating station is located in the North Coast region. 
\subsection{Lower Mainland Region}
Lower mainland region has a total of 1118.7 MW of installed capacity from the 10 integrated Hydroelectric generating stations. Figure~\ref{fig:bclmr} shows the installed capacity contribution of each of these generating stations. 
\begin{figure}[h!]
   \centering 
   \includegraphics[height=2in]{bchlmri}
   \caption{Capacity of BC Hydro Resources in Lower Mainland Region} 
   \label{fig:bclmr}
\end{figure}
\subsection{Vancouver Island Region}
Vancouver Island region has a total of 469.0 MW of installed capacity from the 6 integrated Hydroelectric generating stations. In addition Vancouver Island has a non-integrated Diesel Generating plant operated by BC Hydro with a capacity contribution of 0.3 MW. 
Figure~\ref{fig:bchvir} shows the installed capacity contribution of each of these generating stations. 
\begin{figure}[h!]
\begin{subfigure}[b]{0.5\textwidth}
   \centering 
   \includegraphics[height=2in]{bchvii}
   \caption{Integrated} 
   \label{fig:bchvii}
\end{subfigure}% 
\begin{subfigure}[b]{0.5\textwidth}
   \centering 
   \includegraphics[height=2in]{bchvin}
   \caption{Non-integrated} 
   \label{fig:bchvin}
\end{subfigure}%
   \caption{Capacity of BC Hydro Resources in Vancouver Island} 
   \label{fig:bchvir}
\end{figure}
\subsection{Columbia Region}
Columbia region has a total of 6728.5 MW of installed capacity from the 10 integrated Hydroelectric generating stations. Figure~\ref{fig:bchcr} shows the installed capacity contribution of each of these generating stations. 
\begin{figure}[h!]
   \centering 
   \includegraphics[height=2in]{bchcri}
   \caption{Capacity of BC Hydro Resources in Columbia Region} 
   \label{fig:bchcr}
\end{figure}

\begin{figure}[h!]
\begin{subfigure}[b]{0.5\textwidth}
   \centering 
   \includegraphics[height=2in]{bchpri}
   \caption{Integrated} 
   \label{fig:bchpri}
\end{subfigure}% 
\begin{subfigure}[b]{0.5\textwidth}
   \centering 
   \includegraphics[height=2in]{bchprn}
   \caption{Non-integrated} 
   \label{fig:bchprr}
\end{subfigure}%
   \caption{Capacity of BC Hydro Resources in Peace River Region} 
   \label{fig:capIPP}
\end{figure}
\subsection{Peace River Region}
Peace River region has a total of 4888.4 MW of installed capacity from 5 Generating plants out of which 4 are Hydroelectric and 1 is Themal. 
In addition Peace River Region has 3 non-integrated Diesel Generating plants operated by BC Hydro with a combined capacity contribution of 8.0 MW. 
Figure~\ref{fig:bchprr} shows the installed capacity contribution of each of these generating stations. 
\subsection{North Coast Region}
North Coast region has a total of 53.0 MW of installed capacity from 2 generating stations with Falls River Hydroelectric generating station contributing to 7 MW of capacity and the Thermal generating plant at Prince rupert contributing to 46 MW of installed capacity. Additionally, North Coast has 12 non-integrated Diesel Generating plants and a Hydroelectric generating station operated by BC Hydro with a total combined capacity of 48.8 MW. Figure~\ref{fig:bchncr} shows the installed capacity contribution of each of these generating stations. 
\begin{figure}[h!]
\begin{subfigure}[b]{0.5\textwidth}
   \centering 
   \includegraphics[height=2in]{bchnci}
   \caption{Integrated} 
   \label{fig:bchnci}
\end{subfigure}% 
\begin{subfigure}[b]{0.5\textwidth}
   \centering 
   \includegraphics[height=2in]{bchncn}
   \caption{Non-integrated} 
   \label{fig:bchncn}
\end{subfigure}%
   \caption{Capacity of BC Hydro Resources in North Coast Region} 
   \label{fig:bchncr}
\end{figure}
\section{Conclusion}
This chapter explored BC Hydro heritage resources. The total installed capacity of BCH resources totals to 13314.7 MW, out of which 13257.6 MW is integrated to the BC Hydro system and rest 57 MW is non-integrated. 


\chapter{Independent Power Producer Projects}
\section{Introduction}
As the energy demand grows, there is a need to meet this demand. As BC Hydro heritage resources alone could not meet this demand, BC Hydro acquires power from Independent Power Producers (IPPs) to help meet B.C.'s electricity needs. These IPP projects are developed by power production companies, municipalities, First Nations and customers. IPPs develop and operate projects using energy resources like wind, water and biomass. 
\subsection{Lower Mainland Region}
Lower mainland region has a total of 1202.9 MW of installed capacity from 37 integrated Independent Power Producer Projects. Figure~\ref{fig:lmipp} shows the installed capacity contribution of each of these projects. 
\begin{figure}[h!]
   \centering 
   \includegraphics[height=4.2in]{lmipp}
   \caption{Capacity of Independent Power Producer Projects in Lower Mainland Region} 
   \label{fig:lmipp}
\end{figure}
\subsection{Vancouver Island Region}
Vancouver Island region has a total of 655.0 MW of installed capacity from 20 integrated Independent Power Producer Projects.
Figure~\ref{fig:viipp} shows the installed capacity contribution of each of these generating projects. 

\begin{figure}[h!]
   \centering 
   \includegraphics[height=3.5in]{viipp}
   \caption{Capacity of Independent Power Producer Projects in Vancouver Island} 
   \label{fig:viipp}
\end{figure}

\subsection{Columbia Region}
Columbia region has a total of 1163.5 MW of installed capacity from 28 integrated Independent Power Producer Projects. Figure~\ref{fig:cripp} shows the installed capacity contribution of each of these projects. 
\begin{figure}[h!]
   \centering 
   \includegraphics[height=3.5in]{cripp}
   \caption{Capacity of Independent Power Producer Projects in Columbia Region} 
   \label{fig:cripp}
\end{figure}

\subsection{Peace River Region}
Peace River region has a total of 965.8 MW of installed capacity from 20 integrated Independent Power Producer Projects.
In addition Peace River Region has 1 biomass IPP Project with a capacity of 0.5 MW. 
Figure~\ref{fig:prIPP} shows the installed capacity contribution of each of these projects. 

\begin{figure}[h!]
\begin{subfigure}[b]{0.5\textwidth}
   \centering 
   \includegraphics[height=3.2in]{ncipr}
   \caption{Integrated} 
   \label{fig:ncipr}
\end{subfigure}% 
\begin{subfigure}[b]{0.5\textwidth}
   \centering 
   \includegraphics[width = 3in]{prnipp}
   \caption{Non-integrated} 
   \label{fig:prnipp}
\end{subfigure}%
   \caption{Capacity of Independent Power Producer Projects in Peace River Region} 
   \label{fig:prIPP}
\end{figure}


\subsection{North Coast Region}
North Coast region has a total of 1285 MW of installed capacity from 9 integrated Independent Power Producer Projects.
In addition the region has 4 non-integrated IPP Project with a capacity of 26 MW.  Figure~\ref{fig:ncipp} shows the installed capacity contribution of each of these generating stations. 
\begin{figure}[h!]
\begin{subfigure}{0.5\textwidth}
   \centering 
   \includegraphics[height=2in]{nciipp}
   \caption{Integrated} 
   \label{fig:bchnci}
\end{subfigure}% 
\begin{subfigure}{0.5\textwidth}
   \centering 
   \includegraphics[width=2.8in]{ncnipp}
   \caption{Non-integrated} 
   \label{fig:bchncn}
\end{subfigure}%
   \caption{Capacity of Independent Power Producer Projects in North Coast Region} 
   \label{fig:ncipp}
\end{figure}
\section{Conclusion}
This chapter explored IPP Projects contributing to capacity of BC Hydro System. The total installed capacity of IPP Projects totals to 5298 MW, out of which 5272 MW is integrated to the BC Hydro system and rest 26.5 MW is non-integrated. 

\chapter{ Resources}

\section{Introduction} 
\par BC Hydro meets the demand of its customers by relying on its heritage resources as well as purchasing power from independent power producers (IPPs). A committed supply-side resources are resources for which material regulatory and BC Hydro executive approvals have been secured. 

\subsection{BC Hydro Heritage Resources}
\par BC Hydro relies mainly on the power of moving or falling water to produce mechanical/electrical energy to power the province of B.C. BC Hydro has 31 Hydroelectric and 2 thermal generating stations sending electricity along 75,000 kilometres of power lines that range over mountain tops and river valleys, forested wilderness and high density urban areas. BC Hydro is constantly improving the efficiency and generating power of their facilities, with the ongoing projects including seismic upgrades and modernizing aging equipment.

\par

The integrated and non-integrated resource capacity distribution for BC Hydro heritage resources are as shown in Figure~\ref{fig:capBCH}.


\begin{figure}[h]
\begin{subfigure}[b]{0.5\textwidth}
   \centering 
   \includegraphics[height=2in]{bchi}
   \caption{Integrated} 
   \label{fig:bchi}
\end{subfigure}% 
\begin{subfigure}[b]{0.5\textwidth}
   \centering 
   \includegraphics[height=2in]{bchnon}
   \caption{Non-integrated} 
   \label{fig:bchnon}
\end{subfigure}%
   \caption{Capacity of Integrated BC Hydro Generating Stations by Region} 
   \label{fig:capBCH}
\end{figure}


A detailed breakdown of integrated and non-integrated BC Hydro heritage resources by region and type is given in Table \ref{tab:capBCH} and Table \ref{tab:capBCHNon}. 

 \begin{table}[ht]
\centering 
 \begin{tabular}{ |c|c|c|  }
 \hline
 &  &  \\
 \textbf{Region}&\textbf{Type of Generation}&\textbf{Capacity(MW)}\\ [0.5ex]
 \hline
 & & \\[-1ex]
\textbf{Vancouver Island} & Hydroelectric & 469.0 \\[2ex]

 \multirow{2}{*}{\textbf{Peace Region}}
     & Hydroelectric & 4815.4 \\
     & Thermal & 73.0 \\[2ex]

 \multirow{2}{*}{\textbf{North Coast }}
   & Thermal & 46.0 \\
& Hydroelectric & 7.0 \\[2ex]

\textbf{Lower Mainland} & Hydroelectric & 1118.7 \\[2ex]

\textbf{Columbia Region} & Hydroelectric & 6728.5 \\[2ex]
  \hline
\end{tabular}
 \caption{Capacity of Integrated resources by type and region} 

\label{tab:capBCH}
\end{table}

\begin{table}
 \centering 
 
 \begin{tabular}{ |c|c|c|  }
 \hline
 &  &  \\
 \textbf{Region}&\textbf{Type of Generation}&\textbf{Capacity(MW)}\\ [0.5ex]
 \hline
 & & \\[-1ex]
\textbf{North Coast} & Diesel & 46.8 \\[2ex]
\textbf{Peace Region} & Diesel & 8.0 \\[2ex]
\textbf{North Coast} & Hydroelectric & 2.0 \\[2ex]
\textbf{Vancouver Island} & Diesel & 0.3 \\ [0.5ex]
 \hline
 \end{tabular}
 \caption{Capacity of Non-Integrated resources by type and region} 
\label{tab:capBCHNon}
\end{table}


\subsection{Independent Power Producer Projects}
To support BC's energy needs as they change and grow over time, BC Hydro purchases power from independent power producers (IPPs). IPPs develop and operate projects using resources such as wind, water and biomass and can include power production companies, municipalities, and Indigenous Nations. As of April 1, 2025, BC Hydro has 119 Electricity Purchase Agreements (EPAs) with Independent Power Producers. In total, these projects are capable of delivering approximately 18,662 gigawatt hours of annual supply and approximately 5,298 megawatts of capacity.

The integrated and non-integrated resource capacity distribution for IPP resources by region are as shown in Figure~\ref{fig:capIPP}.

\begin{figure}[h]
\begin{subfigure}[b]{0.5\textwidth}
   \centering 
   \includegraphics[height=2in]{iipp}
   \caption{Integrated} 
   \label{fig:iipp}
\end{subfigure}% 
\begin{subfigure}[b]{0.5\textwidth}
   \centering 
   \includegraphics[height=2in]{ippnon}
   \caption{Non-integrated} 
   \label{fig:ippnon}
\end{subfigure}%
   \caption{Capacity of IPP Resources by Region} 
   \label{fig:capIPP}
\end{figure}

A detailed breakdown of integrated and non-integrated IPP resources by region and type is given in Table \ref{tab:capIPP} and Table \ref{tab:capIPPNon}. 

 \begin{table}[h!]
\centering 
 \begin{tabular}{ |c|l|c|}
 \hline
 &  &  \\
 \textbf{Region}&\textbf{Type of Generation}&\textbf{Capacity(MW)}\\ [0.5ex]
 \hline
 & & \\[-1ex]
\multirow{7}{*}{\textbf{Vancouver Island}}
& Gas Fired Thermal & 275.0 \\
& Wind & 201.2 \\
& Non-Storage Hydro & 93.4 \\
& Biomass & 55.0 \\
& Storage Hydro & 29.9 \\
& Biogas & 0.355 \\
& ERG & 0.3 \\[2ex]

 \multirow{5}{*}{\textbf{Peace Region}}
 & Wind & 515.8 \\
& Biomass & 303.5 \\
& Gas Fired Thermal & 120.0 \\
& Non-Storage Hydro & 20.5 \\
& ERG & 6.0 \\[2ex]

 \multirow{3}{*}{\textbf{North Coast }}
 & Storage Hydro & 934.2 \\
& Non-Storage Hydro & 298.8 \\
& Biomass & 52.0 \\[2ex]

 \multirow{5}{*}{\textbf{Lower Mainland}}
& Non-Storage Hydro & 1046.4 \\
& Biomass & 112.0 \\
& MSW & 24.8 \\
& Storage Hydro & 10.9 \\
& Gas Fired Thermal & 8.8 \\[2ex]

\multirow{7}{*}{\textbf{Columbia Region}}
& Non-Storage Hydro & 509.1 \\
& Storage Hydro & 306.1 \\
& Biomass & 294.9 \\
& Wind & 30.0 \\
& ERG & 16.5 \\
& Biogas & 4.8 \\
& Solar & 2.09 \\[2ex]

\hline
\end{tabular}
\caption{Capacity of Integrated IPP resources by type and region} 
\label{tab:capIPP}
\end{table}

\begin{table}[h!]
 \centering 
 \begin{tabular}{ |c|c|c|}
 \hline
 &  &  \\
 \textbf{Region}&\textbf{Type of Generation}&\textbf{Capacity(MW)}\\ [0.5ex]
 \hline
 & & \\[-1ex]
\textbf{Peace Region} & Biomass & 0.5 \\[2ex]
\multirow{2}{*}{\textbf{North Coast}} 
& Non-Storage Hydro & 15.0 \\
& Storage Hydro & 11.0 \\
 \hline
 \end{tabular}
 \caption{Capacity of Non-Integrated IPP resources by type and region} 
\label{tab:capIPPNon}
\end{table}
\section{Conclusion} 
This chapter explored the existing resources available in B.C. to meet the electricty demand of british columbians. BC Hydro heritage and IPP resources capacities were analysed based on the type and location of the resource.

\end{document}